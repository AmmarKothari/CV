






%%%%%%%%%%%%%%%
%% LANGUAGES %%
%%%%%%%%%%%%%%%

\section
{Languages}
{Languages}
{PDF:Languages}

\BulletItem
English: Native language.

\GapNoBreak
\BulletItem
Spanish: Fluent (speaking, reading, writing).

\GapNoBreak
\BulletItem
Latin: Intermediate (reading); basic (speaking, writing).





%%%%%%%%%%%%%%%%
%% REFERENCES %%
%%%%%%%%%%%%%%%%

\section
{References}
{References}
{PDF:References}

\BulletItem
\textbf{Professor Jonathan Public}
\newline
Professor of Geology and Mechanical Engineering
\newline
First American University
\begin{detail}
\SubItem
1000 First Avenue, Springfield, Massachusetts 22222, USA
\newline
\href{mailto:jonathanpublic@example.com}
{jonathanpublic@example.com}
\,\SubBulletSymbol\,
+1\,(555)\,222-2222
\end{detail}

\Gap
\BulletItem
\textbf{Dr Alice Bob Carol}
\newline
Director, Research \& Development
\newline
Alpha Engineering Firm
\begin{detail}
\SubItem
20 North Street, Oakland, Ohio 33333, USA
\newline
\href{mailto:alicebobcarol@example.com}
{alicebobcarol@example.com}
\,\SubBulletSymbol\,
+1\,(555)\,333-3333
\end{detail}










%%%%%%%%%%%%%%%%%%%%%%%%%%%%%%%%%
%% SECTION WITH USAGE EXAMPLES %%
%%%%%%%%%%%%%%%%%%%%%%%%%%%%%%%%%

\section
{Section\newline
With\newline
Usage\newline
Examples}
{Section With Usage Examples (For PDF Bookmark)}
{PDF:SectionWithUsageExamples:ForPDFLink}

\subsection
{This is a Subsection}
{This is a Subsection}
{PDF:ThisIsASubSection}

\GapNoBreak
\BulletItem
Use \CodeCommand{section} and \CodeCommand{subsection} to create sections and subsections.
These will appear in the PDF bookmarks too.

\GapNoBreak
\BulletItem
This is the second \CodeCommand{BulletItem}.
Long items are automatically indented.
Lorem ipsum dolor sit amet, consectetur adipiscing elit.
Sed sed aliquam massa.
\begin{detail}
\SubBulletItem
This is a \CodeCommand{SubBulletItem}.
Long items are automatically indented.
Lorem ipsum dolor sit amet, consectetur adipiscing elit.
Sed sed aliquam massa.
Aliquam dignissim mi non enim feugiat elementum.
Donec sit amet turpis ac velit ultrices volutpat.
Aliquam vitae elit massa.
\SubBulletItem
This is the second \CodeCommand{SubBulletItem}.
\SubBulletItem
The \CodeCommand{SubBulletItem}'s are between
\CodeCommand{begin\{detail\}} and
\CodeCommand{end\{detail\}} so that they are typeset in a smaller font.
\end{detail}

\Gap
\BulletItem
This is the third \CodeCommand{BulletItem}.

\Gap
\BulletItem
A \CodeCommand{Gap} or \CodeCommand{GapNoBreak} is inserted between the \CodeCommand{BulletItem}'s so that there is a small vertical space between them.
The ``NoBreak'' version prevents page breaking, and should be used to avoid orphaned headings and other formatting issues.

\BigGap
\subsection
{This is the Second Subsection}
{This is the Second Subsection}
{PDF:ThisIsTheSecondSubSection}

\GapNoBreak
\BulletItem
A \CodeCommand{BigGap} or \CodeCommand{BigGapNoBreak} is inserted between subsections so that there is a bigger vertical space between them.
The ``NoBreak'' version prevents page breaking.

%%%%%%%%%%%%%%%%%%%%%%%%%%%%%%%%%%%%%%%%%
%% ANOTHER SECTION WITH USAGE EXAMPLES %%
%%%%%%%%%%%%%%%%%%%%%%%%%%%%%%%%%%%%%%%%%

\section
{Another\newline
Section\newline
With\newline
Usage\newline
Examples}
{Another Section With Usage Examples (For PDF Bookmark)}
{PDF:AnotherSectionWithUsageExamples:ForPDFLink}

\textbf{This is a Plain Heading},
followed by an \CodeCommand{hfill} and a date range
\hfill
\DatestampYM{2015}{10} --
\DatestampYM{2015}{12}

\GapNoBreak
\BulletItem
This is a \CodeCommand{BulletItem}.
\begin{detail}
\SubBulletItem
This is a \CodeCommand{SubBulletItem}.
\end{detail}

\GapNoBreak
\BulletItem
This is a \CodeCommand{BulletItem}.
\begin{detail}
\SubItem
This is a \CodeCommand{SubItem}, which has no bullet.
Note the alignment with the \CodeCommand{BulletItem} above.
\end{detail}

\GapNoBreak
\Item
This is an \CodeCommand{Item}, which has no bullet.
Note the alignment with the \CodeCommand{BulletItem} above.
\begin{detail}
\SubItem
This is a \CodeCommand{SubItem}.
\end{detail}

\GapNoBreak
\NumberedItem{[16]}
This is a \CodeCommand{NumberedItem}.
Note the alignment with the \CodeCommand{SubBulletItem} above.

\GapNoBreak
\NumberedItem{{\CharSpace}[6]}
This is a \CodeCommand{NumberedItem} with a \CodeCommand{CharSpace} in its argument for padding shorter numbers.
Note the alignment with the \CodeCommand{NumberedItem} above.

\BigGap
\textbf{Usage Notes}

\GapNoBreak
\BulletItem
New Lines and Paragraphs
\begin{detail}
\SubBulletItem
To create a new line within the same paragraph (i.e., with the same indentation), use \CodeCommand{newline} instead of \CodeCommand{\textbackslash}.
The latter will not work because it breaks the long table.
\SubBulletItem
To create a new paragraph, use \CodeCommand{par} or simply leave an empty line.
Paragraph indentations (from
\CodeCommand{Item},
\CodeCommand{SubItem},
\CodeCommand{BulletItem},
\CodeCommand{SubBulletItem},
etc.) do not carry across different paragraphs.
\end{detail}

\Gap
\BulletItem
Vertical Spacing Between Items
\begin{detail}
\SubBulletItem
Use \CodeCommand{Gap} or \CodeCommand{GapNoBreak} to insert a small vertical space between items within the same section.
\SubBulletItem
Use \CodeCommand{BigGap} or \CodeCommand{BigGapNoBreak} to insert a bigger vertical space between items within the same section.
\SubBulletItem
The ``NoBreak'' versions prevent page breaking.
\end{detail}

\Gap
\BulletItem
Dates
\begin{detail}
\SubBulletItem
Use
\CodeCommand{DatestampYMD\{YYYY\}\{MM\}\{DD\}},
\CodeCommand{DatestampYM\{YYYY\}\{MM\}}, and
\CodeCommand{DatestampY\{YYYY\}}
to specify dates.
\SubBulletItem
Change the definition of \CodeCommand{DatestampFormatSelection} to adjust how dates are displayed throughout the document (e.g., ``\mbox{2010-12-31}'', ``\mbox{Dec 2010}'', ``\mbox{December 2010}'', ``\mbox{2010}'').
\end{detail}

%%%%%%%%%%%%%%%%%%%%%%%%%%%%%%%%%%%
%% MULTILINGUAL UNICODE EXAMPLES %%
%%%%%%%%%%%%%%%%%%%%%%%%%%%%%%%%%%%

\section
{Multilingual Unicode Examples}
{Multilingual Unicode Examples}
{PDF:MultilingualUnicodeExamples}

\BulletItem
Assortment of unicode characters from
\href{http://www.ltg.ed.ac.uk/~richard/unicode-sample.html}
{http://www.ltg.ed.ac.uk/{\TildeSymbol}richard/unicode-sample.html}

\begin{detail}
\SubItem
\textbf{Latin Extended-A}
Ā ā Ă ă Ą ą Ć ć Ĉ ĉ Ċ ċ Č č Ď ď Đ đ Ē ē Ĕ ĕ Ė ė Ę ę Ě ě Ĝ ĝ Ğ ğ Ġ ġ Ģ ģ Ĥ ĥ Ħ ħ Ĩ ĩ Ī ī Ĭ ĭ Į į İ ı IJ ij Ĵ ĵ
\textbf{Latin Extended-B}
ƀ Ɓ Ƃ ƃ Ƅ ƅ Ɔ Ƈ ƈ Ɖ Ɗ Ƌ ƌ ƍ Ǝ Ə Ɛ Ƒ ƒ Ɠ Ɣ ƕ Ɩ Ɨ Ƙ ƙ ƚ ƛ Ɯ Ɲ ƞ Ɵ Ơ ơ Ƣ ƣ Ƥ ƥ Ʀ Ƨ ƨ Ʃ ƪ ƫ Ƭ ƭ Ʈ Ư ư Ʊ Ʋ Ƴ ƴ Ƶ
\textbf{Latin Extended Additional}
Ḁ ḁ Ḃ ḃ Ḅ ḅ Ḇ ḇ Ḉ ḉ Ḋ ḋ Ḍ ḍ Ḏ ḏ Ḑ ḑ Ḓ ḓ Ḕ ḕ Ḗ ḗ Ḙ ḙ Ḛ ḛ Ḝ ḝ Ḟ ḟ Ḡ ḡ Ḣ ḣ Ḥ ḥ Ḧ ḧ Ḩ ḩ Ḫ ḫ Ḭ ḭ Ḯ ḯ Ḱ ḱ Ḳ ḳ Ḵ ḵ
\textbf{Greek}
ʹ ͵ ͺ ; ΄ ΅ Ά · Έ Ή Ί Ό Ύ Ώ ΐ Α Β Γ Δ Ε Ζ Η Θ Ι Κ Λ Μ Ν Ξ Ο Π Ρ Σ Τ Υ Φ Χ Ψ Ω Ϊ Ϋ ά έ ή ί ΰ α β γ δ ε ζ η θ
\textbf{Cyrillic}
Ё Ђ Ѓ Є Ѕ І Ї Ј Љ Њ Ћ Ќ Ў Џ А Б В Г Д Е Ж З И Й К Л М Н О П Р С Т У Ф Х Ц Ч Ш Щ Ъ Ы Ь Э Ю Я а б в г д е ж з
\textbf{Hebrew}
א ב ג ד ה ו ז ח ט י ך כ ל ם מ ן נ ס ע ף פ ץ צ ק ר ש ת װ ױ ײ ֝ ֞ ֟ ֠ ֡ ֣ ֤ ֥ ֦ ֧ ֨ ֩ ֪ ֫ ֬ ֭ ֮ ֯ ְ ֱ ֒ ֓ ֔
\textbf{Armenian}
{\UseSecondaryFont
Ա Բ Գ Դ Ե Զ Է Ը Թ Ժ Ի Լ Խ Ծ Կ Հ Ձ Ղ Ճ Մ Յ Ն Շ Ո Չ Պ Ջ Ռ Ս Վ Տ Ր Ց Ւ Փ Ք Օ Ֆ ՙ ՚ ՛ ՜ ՝ ՞ ՟ ա բ գ դ ե զ}
\textbf{Thai}
{\UseSecondaryFont
ก ข ฃ ค ฅ ฆ ง จ ฉ ช ซ ฌ ญ ฎ ฏ ฐ ฑ ฒ ณ ด ต ถ ท ธ น บ ป ผ ฝ พ ฟ ภ ม ย ร ฤ ล ฦ ว ศ ษ ส ห ฬ อ ฮ ฯ ะ ั า ำ ิ}
\end{detail}